\setcounter{page}{1}
\setcounter{equation}{0}
\setcounter{figure}{0}
\setcounter{table}{0}

\section*{\'Uvod}
\addcontentsline{toc}{section}{\numberline{}\'Uvod}
	Rozmach smartfónov a tabletov v poslednej dobe zmenil tieto zariadenia na neoddeliteľnú súčasť ľudského života a predmety dennej potreby. Mobilný telefón je jediné zariadenie, ktoré skutočne ľudia nosia neustále so sebou, čo má veľký dopad na potenciál a možnosti možných programov, pre nich určené. Tento poznatok, spolu s fenoménom internetu a dotykovým displejom, ktorý predstavuje skutočne prirodzenú formu interakcie so zariadením, z nich tvorí ideálny prostriedok komunikácie rôzneho druhu.

	Neustály vývoj v hardvérovej oblasti podmieňuje vývoj v oblasti softvérovej do takej miery, že poskytnutý výkon umožňuje využívať takmer všetky z moderných technológií, známych zo sféry stolných počítačov. To spôsobuje možnosť jednoduchého prechodu programátorov na mobilné platformy, respektíve nevyžaduje potrebu osvojovať si úplne nový programovací jazyk, ktorý by bol špeciálne navrhnutý pre prácu v predsa stále hardvérovo ohraničených podmienkach.

	Vývoj na poli dvoj a štvorjadrových ARM procesorov predpovedá blízke, zatiaľ aspoň čiastočné, zmazávanie rozdielov medzi mobilnými a stolnými verziami operačných systémov, kde jasným cieľom tohto snaženia je vytvorenie jedného rozsiahleho ekosystému pre počítače, tablety a smartfóny, a poskytovať tak celistvý používateľský zážitok naprieč všetkými systémami.

	Jedným z dominantných prvkov pôsobiacich na trhu mobilných operačných systémov je firma Google so svojou platformou Android. Otvorenosť a mnohé iné kvality platformy sa pričinili o rozšírenie smartfónov a rozmach systému do takej miery, kde každý druhý smartfón je ovládaný práve týmto operačným systémom.

	Spôsob vytvárania aplikácií pre platformu Android, ktorý sa opiera o všeobecne rozšírené schopnosti dnešných programátorov a hardvérové vymoženosti telefónov, boli dôvodom voľby tejto platformy pre realizáciu cieľa bakalárskej práce.

	Cieľom práce bolo vytvoriť aplikáciu pre platformu Android, ktorá bude svojou funkcionalitou schopná vykonávať inventúru. Inventúra sa bude vykonávať zoskenovaním \nomenclature{QR}{Quick Response - formát kódu rýchlej odozvy} QR kódu pomocou kamery mobilného telefónu, kde bude následne obraz rozpoznaný knižnicou pre dekódovanie QR kódov a obsah kódu bude vrátený aplikácií na spracovanie. Spracované údaje z kódu pomôžu aplikácií zaznačiť tovar ako prítomný, respektíve chýbajúci. Program taktiež ponúka riešenia pre ukladanie a archiváciu výsledkov jednotlivých inventúr.

	Informačné zdroje boli čerpané hlavne z návodov a príručiek z oficiálnej web stránky pre vývojárov pre Android http://developer.android.com. Jednotlivé bežné problémy pri vývoji aplikácie boli riešené a konzultované na diskusnom fóre http://stackoverflow.com, na ktoré prispievajú aj členovia z oficiálneho Android tímu.

	Na začiatku sa práca venuje teórií QR kódov, ich možnostiam, štruktúre a zloženiu. Nasleduje opis vzniku a teória architektúry platformy Android. Neskoršie kapitoly sú venované možnostiam vytvárania aplikácií a nástrojom, ktoré toto umožňujú a prácu uľahčujú. Ďalej sa pojednáva o zložení a štruktúre samotnej aplikácie, hlavných komponentoch	a iných prvkoch skladby. Nasledujúce kapitoly sa venujú analýze a samotnému riešeniu zadanej úlohy, jeho popisu a problémoch pri procese vývoja. V závere práca obsahuje zhrnutie problematiky, zistených problémov a vyskúmaných riešení na tieto problémy.
